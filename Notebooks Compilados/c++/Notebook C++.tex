\documentclass[11pt,letterpaper,twocolumn,twosided]{article}

\usepackage[utf8]{inputenc}
\usepackage[spanish]{babel}
\usepackage{listings}
\usepackage[usenames,dvipsnames]{color}
\usepackage{amsmath}
\usepackage{verbatim}
\usepackage{hyperref}
\usepackage{color}
\usepackage{geometry}

\geometry{verbose,landscape,letterpaper,tmargin=2cm,bmargin=2cm,lmargin=2.5cm,rmargin=1.5cm}

\usepackage{listings}
\usepackage{color}

\definecolor{dkgreen}{rgb}{0,0.6,0}
\definecolor{gray}{rgb}{0.5,0.5,0.5}
\definecolor{mauve}{rgb}{0.58,0,0.82}

\lstset{frame=tb,
  language=C++,
  aboveskip=3mm,
  belowskip=3mm,
  showstringspaces=false,
  columns=flexible,
  basicstyle={\small\ttfamily},
  numbers=none,
  numberstyle=\tiny\color{gray},
  keywordstyle=\color{blue},
  commentstyle=\color{dkgreen},
  stringstyle=\color{mauve},
  breaklines=true,
  breakatwhitespace=true
  tabsize=2
}

\setlength{\columnsep}{0.5in}
\setlength{\columnseprule}{1px}

\begin{document}

\title{Notebook ICPC-UFPS\\Semillero de Investigaci\'on en Linux y Software Libre}
\author{Gerson Yesid L\'azaro - Angie Melissa Delgado}
\maketitle
\tableofcontents
\lstloadlanguages{C++,Java}


\section{Bonus: Input Output}

\subsection{scanf y printf}
\begin{lstlisting}
#include <cstdio>

scanf("%d",&value); //int
scanf("%ld",&value); //long y long int
scanf("%c",&value); //char 
scanf("%f",&value); //float
scanf("%lf",&value); //double
scanf("%s",&value); //char*
scanf("%lld",&value); //long long int
scanf("%x",&value); //int hexadecimal
scanf("%o",&value); //int octal
\end{lstlisting}

\section{Dynamic Programming}

\subsection{Knapsack}
Dados N articulos, cada uno con su propio valor y peso y un tamaño maximo de una mochila, se debe calcular el valor maximo de los elementos que es posible llevar.\\
Debe seleccionarse un subconjunto de objetos, de tal manera que quepan en la mochila y representen el mayor valor posible.

\begin{lstlisting}

#include <algorithm>

const int MAX_WEIGHT = 40;//Peso maximo de la mochila
const int MAX_N = 1000; //Numero maximo de objetos
int N;//Numero de objetos 
int prices[MAX_N];//precios de cada producto
int weights[MAX_N];//pesos de cada producto
int memo[MAX_N][MAX_WEIGHT];//tabla dp

//El metodo debe llamarse con 0 en el id, y la capacidad de la mochila en w
int knapsack(int id, int w) {
  	if (id == N || w == 0) {
  		return 0;
  	}
  	if (memo[id][w] != -1) {
  		return memo[id][w];
  	}
  	if (weights[id] > w){
  		memo[id][w] = knapsack(id + 1, w);
  	}else{
  		memo[id][w] = max(knapsack(id + 1, w), prices[id] + knapsack(id + 1, w - weights[id]));
  	}
  	
  	return memo[id][w];
}

//La tabla memo debe iniciar en -1
memset(memo, -1, sizeof memo);
\end{lstlisting}

\subsection{Longest Increasing Subsequence}
Halla la longitud de la subsecuencia creciente mas larga. MAX debe definirse en el tamaño  limite del array, n es el tamaño del array. Puede aplicarse tambi\'en sobre strings, cambiando el parametro int s[] por string s. Si debe ser estrictamente creciente, cambiar el $\leq $ de "s[j] $\leq $ s[i]" por $<$.

\begin{lstlisting}

const int MAX = 1005;
int memo[MAX];

int longestIncreasingSubsequence(int s[], int n){
	memo[0] = 1;
	int output = 0;
	for (int i = 1; i < n; i++){
		memo[i] = 1;
		for (int j = 0; j < i; j++){
			if (s[j] <= s[i] && memo[i] < memo[j] + 1){
				memo[i] =  memo[j] + 1;
			} 
		}
		if(memo[i] > output){
			output = memo[i];
		}
	}
	return output;
}
\end{lstlisting}

\section{Graph}

\subsection{BFS}
Algoritmo de b\'usqueda en anchura en grafos, recibe un nodo inicial s y visita todos los nodos alcanzables desde s. BFS tambi\'en halla la distancia m\'as corta entre el nodo inicial s y los dem\'as nodos si todas las aristas tienen peso 1.

\begin{lstlisting}

int v, e; //vertices, arcos
int MAX=100005; 
vector<int> ady[MAX]; //lista de Adyacencia
long long distance[MAX];

static void init() {
    for (int j = 0; j <= v; j++) {
        distance[j] = -1;
        ady[j].clear();
    }
}

static void bfs(int s){
    queue<int> q;
    q.push(s); //Inserto el nodo inicial
    distance[s]=0;
    int actual, i, next;
        
    while(q.size()>0){
        actual=q.front();
        q.pop();
        for(i=0; i<ady[actual].size(); i++){
            next=ady[actual][i];
            if(distance[next]==-1){
                distance[next]=distance[actual]+1;
                q.push(next);
            }
        }
    }
}
\end{lstlisting}

\subsection{DFS}
Algoritmo de b\'usqueda en profundidad para grafos. Parte de un nodo inicial s visita a todos sus vecinos. DFS puede ser usado para contar la cantidad de componentes conexas en un grafo y puede ser modificado para que retorne informaci\'on de los nodos dependiendo del problema. Permite hallar ciclos en un grafo.

\begin{lstlisting}

int v, e; //vertices, arcos
int MAX=100005; 
vector<int> ady[MAX];
int marked[MAX];

void init(){
	for (int j = 0; j <= v; j++) {
       marked[j] = 0;
       ady[j].clear();
    }
}

static void dfs(int s){
	marked[s]=1;
	int i, next;

	for(i=0; i<ady[s].size(); i++){
		next=ady[s][i];
		if(marked[next]==0){
			dfs(next);
		}
	}
}
\end{lstlisting}

\subsection{Dijkstra's Algorithm}
Algoritmo que dado un grafo con pesos no negativos halla la ruta m\'inima entre un nodo inicial s y todos los dem\'as nodos.

\begin{lstlisting}

int v,e;
vector<Node> ady[100001];
int marked[100001];
long long distance[100001];
int prev[100001];

class cmp
{
public:
    bool operator()(Node n1,Node n2)
    {
      if(n1.second>n2.second)
      return true;
      else
      return false;
    }
};

void init(){
    long long max=LLONG_MAX;
    for(int j=0; j<=v; j++){
        ady[j].clear();
        marked[j]=0;
        prev[j]=-1;
        distance[j]=max;
    }
}

void dijkstra(int s){
    priority_queue< Node , vector<Node> , cmp > pq;
    pq.push(Node(s, 0));//se inserta a la cola el nodo Inicial.
    distance[s]=0;
    int actual, j, adjacent;
    long long weight;

    while(!pq.empty()){
        actual=pq.top().first;
        pq.pop();
        if(marked[actual]==0){
            marked[actual]=1;
            for(j=0; j<ady[actual].size(); j++){
                adjacent=ady[actual][j].first;
                weight=ady[actual][j].second;
                if(marked[adjacent]==0){
                    if(distance[adjacent] > distance[actual] + weight){
                         distance[adjacent] = distance[actual] + weight;
                         prev[adjacent]=actual;
                         pq.push(Node(adjacent, distance[adjacent]));
                     }
                }
            }
        }
    }
}
\end{lstlisting}

\subsection{Floyd-Warshall's Algorithm}
Algoritmo para grafos que halla la distancia m\'inima entre cualquier par de nodos. Matrix[i][j] guardar\'a la distancia m\'inima entre el nodo i y el j.

\begin{lstlisting}

#define Node pair<int,long long> //(Vertice adyacente, peso)

int v,e;
int matrix[505][505];

void floydWarshall(){
    int k=0;
    int aux, i ,j;
    
    while(k<v){
        for(i=0; i<v; i++){
            if(i!=k){
                for(j=0; j<v; j++){
                    if(j!=k){
                        aux=matrix[i][k]+matrix[k][j];
                        if(aux<matrix[i][j] && aux>0){ 
                            matrix[i][j]=aux;
                        }
                    }
                }
            }
        }
        k++;
    }
}\end{lstlisting}

\subsection{Kruskal's Algorithm}
Algoritmo para hallar el arbol cobertor m\'inimo de un grafo  no dirigido y conexo. Utiliza la t\'ecnica de Union-Find(Conjuntos disjuntos) para detectar que aristas generan ciclos.\\
Para hallar los 2 arboles cobertores minimos, se debe ejecutar el algoritmo v-1 veces, en cada una de ellas dscartar una de las aristas previamente elegidas en el arbol.

\begin{lstlisting}

struct Edge{
    int origen, destino, peso;

    bool operator!=(const Edge& rhs) const{
        if(rhs.origen!=origen || rhs.destino!=destino || rhs.peso!=peso){
            return true;
        }
        return false;
    }
};

int v,e;
int MAX=10001;
int parent[MAX];
int rank[MAX]; 
Edge edges[MAX];
Edge answer[MAX];

void init(){
    for(int i=0; i<v; i++){
        parent[i]=i;
        rank[i]=0;
    }
}

int cmp(const void* a, const void* b)
{   
    struct Edge* a1 = (struct Edge*)a;
    struct Edge* b1 = (struct Edge*)b;
    return a1->peso > b1->peso;
}

int find(int i){
    if(parent[i]!=i){
        parent[i]=find(parent[i]);
    }
    return parent[i];
}

void unionFind(int x, int y){
    int xroot = find(x);
    int yroot = find(y);
 
    // Attach smaller rank tree under root of high rank tree
    // (Union by rank)
    if (rank[xroot] < rank[yroot])
        parent[xroot] = yroot;
    else if (rank[xroot] > rank[yroot])
        parent[yroot] = xroot;
 
    // If ranks are same, then make one as root and increment
    // its rank by one
    else{
        parent[yroot] = xroot;
        rank[xroot]++;
    }
}

void kruskall(){
    Edge actual;
    int aux=0;
    int i=0;
    int x,y;
    qsort(edges, e, sizeof(edges[0]), cmp);

    while(aux<v-1){
        actual=edges[i];
        x=find(actual.origen);
        y=find(actual.destino);

        if(x!=y){
            answer[aux]=actual;
            aux++;
            unionFind(x, y);
        }
        i++;
    }
}
\end{lstlisting}

\subsection{Tarjan's Algorithm}


\begin{lstlisting}

vector<int> ady[1010];
int marked[1010];
int prev[1010];
int dfs_low[1010];
int dfs_num[1010];
int itsmos[1010];
int n, e;
int dfsRoot,rootChildren,cont;
vector<pair<int,int>> bridges;

void init(){
    bridges.clear();
    cont=0;
    int i;
    for(i=0; i<n; i++){
        ady[i].clear();
        marked[i]=0;
        prev[i]=-1;
        itsmos[i]=0;
    }
}

void dfs(int u){
    dfs_low[u]=dfs_num[u]=cont;
    cont++;
    marked[u]=1; //esto no estaba >.<
    int j, v;

    for(j=0; j<ady[u].size(); j++){
        v=ady[u][j];
        if(marked[v]==0){
            prev[v]=u;
            //para el caso especial 
            if(u==dfsRoot){
                rootChildren++;
            }
            dfs(v);
            //PARA ITSMOS
            if(dfs_low[v]>=dfs_num[u]){
                itsmos[u]=1;
            }
            //PARA bridges
            if(dfs_low[v]>dfs_num[u]){
                bridges.push_back(make_pair(min(u,v),max(u,v)));
            }
            dfs_low[u]=min(dfs_low[u], dfs_low[v]);
        }else if(v!=prev[u]){ //Arco que no sea backtrack
            dfs_low[u]=min(dfs_low[u], dfs_num[v]);
        }
    }
}
\end{lstlisting}

\section{Math}

\subsection{Binary Exponentiation}
Realiza $a^{b}$ y retorna el resultado m\'odulo c. Si se elimina el m\'odulo c, debe tenerse precauci\'on para no exceder el l\'imite

\begin{lstlisting}


int binaryExponentiation(int a, int b, int c){
    if (b == 0){
    	return 1;
    } 
    if (b % 2 == 0){
        int temp = binaryExponentiation(a,b/2, c);
        return ((long long)(temp) * temp) % c;
    }else{
        int temp = binaryExponentiation(a, b-1, c);
        return ((long long)(temp) * a) % c;
    }
}
\end{lstlisting}

\subsection{Binomial Coefficient}
Calcula el coeficiente binomial nCr, entendido como el n\'umero de subconjuntos de k elementos escogidos de un conjunto con n elementos.

\begin{lstlisting}

long long binomialCoefficient(long long n, long long r) {
  if (r < 0 || n < r) { 
  	return 0; 
  }
  r = min(r, n - r);
  long long ans = 1;
  for (int i = 1; i <= r; i++) {
    ans = ans * (n - i + 1) / i;
  }
  return ans;
}
\end{lstlisting}

\subsection{Catalan Number}
Guarda en el array Catalan Numbers los numeros de Catalan hasta MAX.

\begin{lstlisting}

const int MAX = 30;
long long catalanNumbers[MAX+1];

void catalan(){
	catalanNumbers[0] = 1;
	for(int i = 1; i <= MAX; i++){
		catalanNumbers[i] = (long long)(catalanNumbers[i-1]*((double)(2*((2 * i)- 1))/(i + 1)));
	}
}
\end{lstlisting}

\subsection{Euler Totient}
Funci\'on totient o indicatriz ($\phi$) de Euler. Para cada posici\'on n del array result retorna el n\'umero de enteros positivos menores o iguales a n que son coprimos con n (Coprimos: MCD=1)

\begin{lstlisting}

#include <string.h>

const int MAX = 100;
int result[MAX]; 

void totient () {
	bool temp[MAX];
	int i,j;
	memset(temp,1,sizeof(temp));
	for(i = 0; i < MAX; i++) {
		result[i] = i;
	}
	for(i = 2; i < MAX; i++){
		if(temp[i]) {
			for(j = i; j < MAX ; j += i){
				temp[j] = false;
				result[j] = result[j] - (result[j]/i) ;
			}
			temp[i] = true ;
		}
	}
}
\end{lstlisting}

\subsection{Gaussian Elimination}
Resuelve sistemas de ecuaciones lineales por eliminaci\'on Gaussiana.  matrix contiene los valores de la matriz cuadrada y result los resultados de las ecuaciones. Retorna un vector con el valor de las n incongnitas. Los resultados pueden necesitar redondeo.

\begin{lstlisting}

#include <vector>
#include <algorithm>
#include <limits>
#include <cmath>

const int MAX = 100;
int n = 3;
double matrix[MAX][MAX];
double result[MAX];

vector<double> gauss() {
	
  	vector<double> ans(n, 0);
  	double temp;
	for (int i = 0; i < n; i++) {
    	int pivot = i;
	    for (int j = i + 1; j < n; j++) {
	    	temp = fabs(matrix[j][i]) - fabs(matrix[pivot][i]);
	      	if (temp > numeric_limits<double>::epsilon()) {
	        	pivot = j;
	      	}
	    }
	    
	    swap(matrix[i], matrix[pivot]);
	    swap(result[i], result[pivot]);
	    
	    if (!(fabs(matrix[i][i]) < numeric_limits<double>::epsilon())) {
	    	
	    	for (int k = i + 1; k < n; k++) {
		      	temp = -matrix[k][i] /  matrix[i][i];
		      	matrix[k][i] = 0;
		      	for (int l = i + 1; l < n; l++) {
		        	matrix[k][l] += matrix[i][l] * temp;
		      	}
		      	result[k] += result[i] * temp;
		    }
	    }
  	}
  	for (int m = n - 1; m >= 0; m--) {
    	temp = result[m];
    	for (int i = n - 1; i > m; i--) {
    		temp -= ans[i] * matrix[m][i];
    	}
    	ans[m] = temp / matrix[m][m];
  	}
  	return ans;
}
\end{lstlisting}

\subsection{Greatest common divisor}
Calcula el m\'aximo com\'un divisor entre a y b mediante el algoritmo de Euclides

\begin{lstlisting}
int mcd(int a, int b) {
	int aux;
	while(b!=0){
		a %= b;
		aux = b;
		b = a;
		a = aux;
	}
	return a;
}
\end{lstlisting}

\subsection{Lowest Common Multiple}
Calculo del m\'inimo com\'un m\'ultiplo usando el m\'aximo com\'un divisor. REQUIERE mcd(a,b)

\begin{lstlisting}

int mcm(int a, int b) {
	return a*b/mcd(a,b);
}
\end{lstlisting}

\subsection{Prime Factorization}
Guarda en primeFactors la lista de factores primos del value de menor a mayor. \\IMPORTANTE: Debe ejecutarse primero la criba de Eratostenes.  La criba debe existir al menos hasta la raiz cuadrada de value (se  recomienda dejar un poco de excedente).

\begin{lstlisting}

#include <vector>

vector <long long> primeFactors;

void calculatePrimeFactors(long long value){
	primeFactors.clear();
	long long temp = value;
	int factor;
	for (int i = 0; (long long)primes[i] * primes[i] <= value; ++i){
		factor = primes[i];
		while (temp % factor == 0){
			primeFactors.push_back(factor);
			temp /= factor;
		}
	}
	if (temp != 1) {
		primeFactors.push_back(temp);
	}
}
\end{lstlisting}

\subsection{Sieve of Eratosthenes}
Guarda en primes los n\'umeros primos menores o iguales a MAX

\begin{lstlisting}

#include <vector>

const int MAX = 10000000;
vector<int> primes;
bool sieve[MAX+5];

void calculatePrimes() {
  sieve[0] = sieve[1] = 1;
  int i;
  for (i = 2; i * i <= MAX; i++) {
    if (!sieve[i]) {
      primes.push_back(i);
      for (int j = i * i; j <= MAX; j += i)
        sieve[j] = true;
    }
  }
  for(;i <= MAX; i++){
  	if (!sieve[i]) {
      primes.push_back(i);
  	}
  }
}
\end{lstlisting}

\section{String}

\subsection{KMP's Algorithm}
Encuentra si el string pattern se encuentra en el string cadena.

\begin{lstlisting}

#include <vector>

vector<int> table(string pattern){
	int m=pattern.size();
	vector<int> border(m);
	border[0]=0;

	for(int i=1; i<m; ++i){
		border[i]=border[i-1];
		while(border[i]>0 && pattern[i]!=pattern[border[i]]){
			border[i]=border[border[i]-1];
		}
		if(pattern[i] == pattern[border[i]]){
			border[i]++;
		}
	}
	return border;
}

bool kmp(string cadena, string pattern){
	int n=cadena.size();
	int m=pattern.size();
	vector<int> tab=table(pattern);
	int seen=0;

	for(int i=0; i<n; i++){
		while(seen>0 && cadena[i]!=pattern[seen]){
			seen=tab[seen-1];
		}
		if(cadena[i]==pattern[seen])
			seen++;
		if(seen==m){
			return true;
		}
	}
	return false;
}
\end{lstlisting}

\section{Tips and formulas}

\subsection{Catalan Number}

{\LARGE $C_{n} = \frac{1}{n+1}\binom{2n}{n} = \frac{(2n)!}{(n+1)!n!}$}\\ \\
Primeros 30 n\'umeros de Catal\'an:\\
\begin{tabular}{|l|l|}
\hline
n & $C_{n}$ \\ \hline
0 & 1 \\ \hline
1 & 1 \\ \hline
2 & 2 \\ \hline
3 & 5 \\ \hline
4 & 14 \\ \hline
5 & 42 \\ \hline
6 & 132 \\ \hline
7 & 429 \\ \hline
8 & 1.430\\ \hline
9 & 4.862\\ \hline
10 & 16.796\\ \hline
11 & 58.786\\ \hline
12 & 208.012\\ \hline
13 & 742.900\\ \hline
14 & 2.674.440\\ \hline
15 & 9.694.845\\ \hline
16 & 35.357.670\\ \hline
17 & 129.644.790\\ \hline
18 & 477.638. 700\\ \hline
19 & 1.767.263.190\\ \hline
20 & 6.564.120.420\\ \hline
21 & 24.466.267.020\\ \hline
22 & 91.482.563.640\\ \hline
23 & 343.059.613.650\\ \hline
24 & 1.289.904.147.324\\ \hline
25 & 4.861.946.401.452\\ \hline
26 & 18.367.353.072.152\\ \hline
27 & 69.533.550.916.004\\ \hline
28 & 263.747.951.750.360\\ \hline
29 & 1.002.242.216.651.368\\ \hline
30 & 3.814.986.502.092.304\\ \hline
\end{tabular}



\subsection{Euclidean Distance}
F\'ormula para calcular la distancia Euclideana entre dos puntos en el plano cartesiano (x,y). \\
Extendible a 3 dimensiones\\ \\
$ d_{E}(P_{1},P_{2}) = \sqrt{(x_{2}-x_{1})^{2}+(y_{2}-y_{1})^{2}} $




\subsection{Permutation and combination}

Combinaci\'on (Coeficiente Binomial): N\'umero de subconjuntos de k elementos escogidos de un conjunto con n elementos\\ \\
{\LARGE $ \binom{n}{k} = \binom{n}{n-k} = \frac{n!}{k!(n-k)!} $}\\

Combinaci\'on con repetici\'on: N\'umero de grupos formados por n elementos, partiendo de m tipos de elementos.\\ \\


{\LARGE $ CR_{m}^{n} = \binom{m+n-1}{n} = \frac{(m + n - 1)!}{n!(m-1)!} $}\\

Permutaci\'on: N\'umero de formas de agrupar n elementos, donde importa el orden y sin repetir elementos\\ \\

{\LARGE $ P_{n} = n! $}\\

Elegir r elementos de n posibles con repetici\'on\\ \\

{\LARGE $ n^{r} $}\\

Permutaciones con repetici\'on: Se tienen n elementos donde el primer elemento se repite a veces , el segundo b veces , el tercero c veces, ...\\ \\

{\LARGE $ PR_{n}^{a,b,c...} = \frac{P_{n}}{a!b!c!...}$}\\

Permutaciones sin repetici\'on: N\'umero de formas de agrupar r elementos de n disponibles, sin repetir elementos \\ \\

{\LARGE $ \frac{n!}{(n-r)!}$}\\
}
\subsection{Time Complexities}

Aproximaci\'on del mayor n\'umero n de datos que pueden procesarse para cada una de las complejidades algoritmicas. Tomar esta tabla solo como referencia.\\ 

\begin{tabular}{|l|l|}
\hline
Complexity & n\\ \hline
$O(n!)$ & 11\\ \hline
$O(n^{5})$ & 50\\ \hline
$O(2^{n}*n^{2})$ & 18\\ \hline
$O(2^{n}*n)$ & 22\\ \hline
$O(n^{4})$ & 100\\ \hline
$O(n^{3})$ & 500\\ \hline
$O(n^{2}\log_{2}n)$ & 1.000\\ \hline
$O(n^{2})$ & 10.000\\ \hline
$O(n\log_{2}n)$ & $10^{6}$\\ \hline
$O(n)$ & $10^{8}$\\ \hline
$O(\sqrt{n})$ & $10^{16}$\\ \hline
$O(\log_{2}n)$ & -\\ \hline
$O(1)$ & -\\ \hline
\end{tabular}


\subsection{mod: properties}

\begin{enumerate}
\item (a \% b) \% b = a \% b (Propiedad neutro)
\item (ab) \% c = ((a \% c)(b \% c)) \% c (Propiedad asociativa en multiplicaci\'on)
\item (a + b) \% c = ((a \% c) + (b \% c)) \% c (Propiedad asociativa en suma)
\end{enumerate}
\end{document}
