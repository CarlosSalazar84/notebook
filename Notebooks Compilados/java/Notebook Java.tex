\documentclass[11pt,letterpaper,twocolumn,twosided]{article}

\usepackage[utf8]{inputenc}
\usepackage[spanish]{babel}
\usepackage{listings}
\usepackage[usenames,dvipsnames]{color}
\usepackage{amsmath}
\usepackage{verbatim}
\usepackage{hyperref}
\usepackage{color}
\usepackage{geometry}

\geometry{verbose,landscape,letterpaper,tmargin=2cm,bmargin=2cm,lmargin=2.5cm,rmargin=1.5cm}

\usepackage{listings}
\usepackage{color}

\definecolor{dkgreen}{rgb}{0,0.6,0}
\definecolor{gray}{rgb}{0.5,0.5,0.5}
\definecolor{mauve}{rgb}{0.58,0,0.82}

\lstset{frame=tb,
  language=C++,
  aboveskip=3mm,
  belowskip=3mm,
  showstringspaces=false,
  columns=flexible,
  basicstyle={\small\ttfamily},
  numbers=none,
  numberstyle=\tiny\color{gray},
  keywordstyle=\color{blue},
  commentstyle=\color{dkgreen},
  stringstyle=\color{mauve},
  breaklines=true,
  breakatwhitespace=true
  tabsize=2
}

\setlength{\columnsep}{0.5in}
\setlength{\columnseprule}{1px}

\begin{document}

\title{Notebook ICPC-UFPS\\Semillero de Investigaci\'on en Linux y Software Libre}
\author{Gerson Yesid L\'azaro - Angie Melissa Delgado}
\maketitle
\tableofcontents
\lstloadlanguages{C++,Java}


\section{Bonus: Input Output}

\subsection{Scanner}
Libreria para recibir las entradas; reemplaza el Scanner original, mejorando su eficiencia. \\
Contiene los metodos next, nextLine y hasNext. Para recibir datos numericos, hacer casting de next.

\begin{lstlisting}


import java.io.BufferedReader;
import java.io.IOException;
import java.io.InputStreamReader;
import java.util.StringTokenizer;

public class Main {

    static class Scanner{
        InputStreamReader isr = new InputStreamReader(System.in);
        BufferedReader br = new BufferedReader(isr);
        StringTokenizer st = new StringTokenizer("");
        int espacios = 0;

        public String nextLine() throws IOException{
        	if(espacios>0){
            	espacios--;
            	return "";
            }else if(st.hasMoreTokens()){
                StringBuilder salida = new StringBuilder();
                while(st.hasMoreTokens()){
                    salida.append(st.nextToken());
                    if(st.countTokens()>0){
                        salida.append(" ");
                    }
                }
                return salida.toString();
            }
            return br.readLine();
        }

        public String next() throws IOException{
        	espacios=0;
            while (!st.hasMoreTokens() ) {
                st = new StringTokenizer(br.readLine() );
            }
            return st.nextToken();
        }

        public boolean hasNext() throws IOException{
            while (!st.hasMoreTokens()) {
                String linea = br.readLine();
                if (linea == null) {
                    return false;
                }
                if(linea.equals("")){
                	espacios++;
                }
                st = new StringTokenizer(linea);
            }
            return true;
        }
    }
}
\end{lstlisting}

\subsection{printWriter}
Utilizar en lugar del System.out.println para mejorar la eficiencia.
\begin{lstlisting}


import java.io.PrintWriter;

PrintWriter so = new PrintWriter(System.out);
so.print("Imprime sin salto de linea");
so.println("Imprime con salto de linea");

//Al finalizar
bw.flush();
\end{lstlisting}

\section{Dynamic Programming}
Dados N articulos, cada uno con su propio valor y peso y un tama\~no maximo de una mochila, se debe calcular el valor maximo de los elementos que es posible llevar.\\
Debe seleccionarse un subconjunto de objetos, de tal manera que quepan en la mochila y representen el mayor valor posible.

\subsection{Knapsack}
\begin{lstlisting}


static int MAX_WEIGHT = 40;//Peso maximo de la mochila
static int MAX_N = 1000; //Numero maximo de objetos
static int N;//Numero de objetos 
static int prices[] = new  int[MAX_N];//precios de cada producto
static int weights[] = new int[MAX_N];//pesos de cada producto
static int memo[][]= new int[MAX_N][MAX_WEIGHT];//tabla dp

//El metodo debe llamarse con 0 en el id, y la capacidad de la mochila en w
static int knapsack(int id, int w) {
  	if (id == N || w == 0) {
  		return 0;
  	}
  	if (memo[id][w] != -1) {
  		return memo[id][w];
  	}
  	if (weights[id] > w){
  		memo[id][w] = knapsack(id + 1, w);
  	}else{
  		memo[id][w] = Math.max(knapsack(id + 1, w), prices[id] + knapsack(id + 1, w - weights[id]));
  	}
  	return memo[id][w];
}

//Antes de llamar al metodo, todos los campos de la tabla memo deben iniciarse a -1	
\end{lstlisting}

\subsection{Longest increasing subsequence}
Halla la longitud de la subsecuencia creciente mas larga. MAX debe definirse en el tama\~no  limite del array, n es el tama\~no del array. Puede aplicarse tambi\'en sobre strings, cambiando el parametro int s[] por string s. Si debe ser estrictamente creciente, cambiar el <= de s[j] <= s[i] por <

\begin{lstlisting}

	static int MAX = 1005;
	static int memo[] = new int[MAX];
	
	static int longestIncreasingSubsequence(int s[]){
		int n = s.length;
		memo[0] = 1;
		int output = 0;
		for (int i = 1; i < n; i++){
			memo[i] = 1;
			for (int j = 0; j < i; j++){
				if (s[j] <= s[i] && memo[i] < memo[j] + 1){
					memo[i] =  memo[j] + 1;
				} 
			}
			if(memo[i] > output){
				output = memo[i];
			}
		}
		return output;
	}
\end{lstlisting}

\section{Graphs}

\subsection{BFS}
Algoritmo de b\'usqueda en anchura en grafos, recibe un nodo inicial s y visita todos los nodos alcanzables desde s. BFS tambi\'en halla la distancia m\'as corta entre el nodo inicial s y los dem\'as nodos si todas las aristas tienen peso 1.

\begin{lstlisting}

import java.util.ArrayList;
import java.util.Queue;
import java.util.LinkedList;


static int v, e; //vertices, arcos
static int MAX=100005; 
static ArrayList<Integer> ady[] = new ArrayList[MAX]; //lista de Adyacencia
static long distance[] = new long[MAX];

//M\'etodo para limpiar los valores de las estructuras.
static void init() {
    for (int j = 0; j <= v; j++) {
        distance[j] = -1;
        ady[j] = new ArrayList<Integer>();
    }
}

//Recibe el nodo inicial s
static void bfs(int s){
    Queue<Integer> q=new LinkedList<Integer>();
    q.add(s); //Inserto el nodo inicial
    distance[s]=0;
    int actual, i, next;
        
    while(!q.isEmpty()){
        actual=q.poll();
        for(i=0; i<ady[actual].size(); i++){
            next=ady[actual].get(i);
            if(distance[next]==-1){
                distance[next]=distance[actual]+1;
                q.add(next);
            }
        }
    }
}\end{lstlisting}

\subsection{DFS}
Algoritmo de b\'usqueda en profundidad para grafos. Parte de un nodo inicial s visita a todos sus vecinos. DFS puede ser usado para contar la cantidad de componentes conexas en un grafo y puede ser modificado para que retorne informaci\'on de los nodos dependiendo del problema. Permite hallar ciclos en un grafo.

\begin{lstlisting}

import java.util.ArrayList;

static int v, e; //vertices, arcos
static int MAX=100005; 
static ArrayList<Integer> ady[] = new ArrayList[MAX];
static boolean marked[] = new boolean[MAX];

//Limpia las estrucuturas de datos
static void init(){
	 for (int j = 0; j <= v; j++) {
        marked[j] = false;
        ady[j] = new ArrayList<Integer>();
    }
}

//Recibe el nodo inicial s
static void dfs(int s){
	marked[s]=true;
	int i, next;

	for(i=0; i<ady[s].size(); i++){
		next=ady[s].get(i);
		if(!marked[next]){
			dfs(next);
		}
	}
}
\end{lstlisting}

\subsection{Dijkstra's Algorithm}
Algoritmo que dado un grafo con pesos no negativos halla la ruta m\'inima entre un nodo inicial s y todos los dem\'as nodos.

\begin{lstlisting}

import java.util.ArrayList;
import java.util.PriorityQueue;


static int v, e; //vertices, arcos
static int MAX=100005; 
static ArrayList<Node> ady[] = new ArrayList[MAX];
static int marked[] = new int[MAX];
static long distance[] = new long[MAX];
static int prev[] = new int[MAX];

//M\'etodo para limpiar los valores de las estructuras.
//Llamarlo siempre antes de utilizar el m\'etodo dijkstra()
static void init() {
    long max = Long.MAX_VALUE;

    for (int j = 0; j <= v; j++) {
        marked[j] = 0;
        prev[j] = -1;
        distance[j] = max;
        ady[j] = new ArrayList<Node>();
    }
}

    
//Recibe el nodo inicial s
static void dijkstra(int s) {
    PriorityQueue<Node> pq=new PriorityQueue<Node>();
    pq.add(new Node(s, 0));//se inserta a la cola el nodo Inicial.
    distance[s] = 0;
    int actual, j, adjacent;
    long weight;
    Node x;

    while (pq.size() > 0) {
        actual = pq.peek().adjacent;
        if (marked[actual] == 0) {
            marked[actual] = 1;
            for (j = 0; j < ady[actual].size(); j++) {
                adjacent = ady[actual].get(j).adjacent;
                weight = ady[actual].get(j).cost;
                if (marked[adjacent] == 0) {
                    if (distance[adjacent] > distance[actual] + weight) {
                        distance[adjacent] = distance[actual] + weight;
                        prev[adjacent] = actual;
                        pq.add(new Node(adjacent, distance[adjacent]));
                    }
                }
            }
        }
        pq.poll();
    }
}

//Retorna en un String la ruta desde s hasta t
//Recibe el nodo destino t
static String path(int t) {
    String r="";
    while(prev[t]!=-1){
        r="-"+t+r;
        t=prev[t];
    }
    if(t!=-1){
        r=t+r;
    }
    return r;
}


static class Node implements Comparable<Node> {

    public int adjacent;
    public long cost;

    public Node(int ady, long c) {
        this.adjacent = ady;
        this.cost = c;
    }

    @Override
    public int compareTo(Node o) {
        if (this.cost >= o.cost) {
            return 1;
        } else {
            return -1;
        }
    }
}    

\end{lstlisting}

\subsection{Floyd-Warshall's Algorithm}
Algoritmo para grafos que halla la distancia m\'inima entre cualquier par de nodos. Matrix[i][j] guardar\'a la distancia m\'inima entre el nodo i y el j.\\
Ajustar los tipos de datos segun el problema.

\begin{lstlisting}

static int v, e; //vertices, arcos
static int MAX=505;     
static int matrix[][]=new int [MAX][MAX];

//M\'etodo para limpiar las estructuras de datos
static void init() {
    int i, j;
    for(i=0; i<v; i++){
        for(j=0; j<v; j++){
            matrix[i][j]=-1;
        }
    }
}

static void floydWarshall(){
    int i,j,k, aux;
    k=0;
    while(k<v){
        for(i=0; i<v; i++){
            if(i!=k){
                for(j=0; j<v; j++){
                    if(j!=k){
                        aux=matrix[i][k]+matrix[k][j];
                        if(aux<matrix[i][j] && aux>0){ 
                            matrix[i][j]=aux;
                        }
                    }
                }
            }
        }
        k++;
    }
}
\end{lstlisting}

\subsection{Kruskal's Algorithm}
Algoritmo para hallar el arbol cobertor m\'inimo de un grafo  no dirigido y conexo. Utiliza la t\'ecnica de Union-Find(Conjuntos disjuntos) para detectar que aristas generan ciclos.

\begin{lstlisting}

import java.util.ArrayList;
import java.util.Collections;


static int v, e; //vertices, arcos
static int MAX=100005;     
static int parent[]= new int [MAX];
static int rank[]= new int [MAX]; 
static ArrayList<Edge> edges;
static ArrayList<Edge> answer;

//limpiar las estructuras de datos    
static void init() {
    edges=new ArrayList<Edge>();
    answer=new ArrayList<Edge>();
    for (int j = 0; j <= v; j++) {
        parent[j] = j;
        rank[j] = 0;
    }
}

//UNION-FIND   
static int find(int i){
    if(parent[i]!=i){
        parent[i]=find(parent[i]);
    }
        return parent[i];
}

static void unionFind(int x, int y){
    int xroot = find(x);
    int yroot = find(y);
     
    if (rank[xroot] < rank[yroot])
        parent[xroot] = yroot;
    else if (rank[xroot] > rank[yroot])
        parent[yroot] = xroot;
     
    else{
        parent[yroot] = xroot;
        rank[xroot]++;
    }
}

static void kruskall(){
    Edge actual;
    int aux=0;
    int i=0;
    int x,y;
    Collections.sort(edges);
        
    while(aux<(v-1)){
        actual=edges.get(i);
        x=find(actual.src);
        y=find(actual.dest);

        if(x!=y){
            answer.add(actual);
            aux++;
            unionFind(x, y);
        }
        i++;
    }
}

static class Edge implements Comparable<Edge> {

    public int src, dest, weight;
    
    public Edge(int s, int d, int w) {
        this.src = s;
        this.dest = d;
        this.weight=w;
    }
    
    @Override
    public int compareTo(Edge o) {
        return this.weight-o.weight;
    }
}
\end{lstlisting}

\subsection{Trajan's Algorithm}
Algoritmo para hallar los puentes e itsmos en un grafo no dirigido.

\begin{lstlisting}

import java.util.ArrayList;
import java.lang.Math;


static int n, e; //vertices, arcos
static int MAX=1010;     
static ArrayList<Integer> ady[]=new ArrayList [MAX];
static boolean marked[]=new boolean [MAX];
static int prev[]=new int [MAX];
static int dfs_low[]=new int [MAX];
static int dfs_num[]=new int [MAX];
static int itsmos[]=new int [MAX];
static ArrayList<Edge> bridges;
static int dfsRoot, rootChildren, cont;

static void init() {
    bridges=new ArrayList<Edge>();
    cont=0;
    int i;
    for(i=0; i<n; i++){
        ady[i]=new ArrayList<Integer>();
        marked[i]=false;
        prev[i]=-1;
        itsmos[i]=0;
    }
}

static void dfs(int u){
    dfs_low[u]=cont;
    dfs_num[u]=cont;
    cont++;
    marked[u]=true;
    int j, v;

    for(j=0; j<ady[u].size(); j++){
        v=ady[u].get(j);
        if(!marked[v]){
            prev[v]=u;
            //Caso especial 
            if(u==dfsRoot){
                rootChildren++;
            }
            dfs(v);
            //PARA ITSMOS
            if(dfs_low[v]>=dfs_num[u]){
                itsmos[u]=1;
            }
            //PARA PUENTES
            if(dfs_low[v]>dfs_num[u]){
                bridges.add(new Edge(Math.min(u,v),Math.max(u,v)));
            }
            dfs_low[u]=Math.min(dfs_low[u], dfs_low[v]);
        }else if(v!=prev[u]){ //Arco que no sea backtrack
            dfs_low[u]=Math.min(dfs_low[u], dfs_num[v]);
        }
    }
}

static class Edge{

    public int src, dest;
    
    public Edge(int s, int d) {
        this.src = s;
        this.dest = d;
    }
}

\end{lstlisting}

\section{Math}

\subsection{Binary Exponentiation}
Realiza $a^{b}$ y retorna el resultado m\'odulo c. Si se elimina el m\'odulo c, debe tenerse precauci\'on para no exceder el l\'imite

\begin{lstlisting}

static int binaryExponentiation(int a, int b, int c){
    if (b == 0){
    	return 1;
    } 
    if (b % 2 == 0){
        int temp = binaryExponentiation(a,b/2, c);
        return (int)(((long)(temp) * temp) % c);
    }else{
        int temp = binaryExponentiation(a, b-1, c);
        return (int)(((long)(temp) * a) % c);
    }
}
\end{lstlisting}

\subsection{Binomial Coefficient}
Calcula el coeficiente binomial nCr, entendido como el n\'umero de subconjuntos  de k elementos escogidos de un conjunto con n elementos.

\begin{lstlisting}

static long binomialCoefficient(long n, long r) {
  if (r < 0 || n < r) { 
  	return 0; 
  }
  r = Math.min(r, n - r);
  long ans = 1;
  for (int i = 1; i <= r; i++) {
    ans = ans * (n - i + 1) / i;
  }
  return ans;
}
\end{lstlisting}

\subsection{Catalan Number}
Guarda en el array Catalan Numbers los numeros de Catalan hasta MAX.

\begin{lstlisting}

static int MAX = 30;
static long catalanNumbers[] = new long[MAX+1];

static void catalan(){
	catalanNumbers[0] = 1;
	for(int i = 1; i <= MAX; i++){
		catalanNumbers[i] = (long)(catalanNumbers[i-1]*((double)(2*((2 * i)- 1))/(i + 1)));
	}
}
\end{lstlisting}

\subsection{Euler Totient}
Funci\'on totient o indicatriz ($\phi $) de Euler. Para cada posici\'on n del array result retorna el n\'umero de enteros positivos menores o iguales a n que son coprimos con n (Coprimos: MCD=1)

\begin{lstlisting}

static void totient(int n, int resultados[]){
	boolean aux[]=new boolean[n];
	for(int i=0; i<n; i++) {
		resultados[i]=i;
	}
	for(int i=2; i<n; i++){
		if(!aux[i]) {
			for(int j=i; j<n ; j+=i){
				aux[j]=true;
				resultados[j]= resultados[j]-(resultados[j]/i) ;
			}
		aux[i] = false;
		}
	}
}
\end{lstlisting}

\subsection{Gaussian Elimination}
Resuelve sistemas de ecuaciones lineales por eliminaci\'on Gaussiana. matrix contiene los valores de la matriz cuadrada y result los resultados de las ecuaciones. Retorna un vector con el valor de las n incongnitas. Los resultados pueden necesitar redondeo.

\begin{lstlisting}

import java.util.ArrayList;

static int MAX = 100;
static int n = 3;
static double matrix[][] = new double[MAX][MAX];
static double result[] = new double[MAX];

static ArrayList<Double> gauss() {
	
  	ArrayList<Double> ans = new ArrayList<Double>();
  	for(int i=0;i<n;i++){
  		ans.add(0.0);
  	}
  	double temp;
	for (int i = 0; i < n; i++) {
    	int pivot = i;
	    for (int j = i + 1; j < n; j++) {
	    	temp = Math.abs(matrix[j][i]) - Math.abs(matrix[pivot][i]);
	      	if (temp > 0.000001) {
	        	pivot = j;
	      	}
	    }
	    double temp2[] = new double[n];
	    System.arraycopy(matrix[i],0,temp2,0,n);
	    System.arraycopy(matrix[pivot],0,matrix[i],0,n);
	    System.arraycopy(temp2,0,matrix[pivot],0,n);
	    temp = result[i];
	    result[i] = result[pivot];
	    result[pivot] = temp;
	    
	    if (!(Math.abs(matrix[i][i]) < 0.000001)) {
	    	
	    	for (int k = i + 1; k < n; k++) {
		      	temp = -matrix[k][i] /  matrix[i][i];
		      	matrix[k][i] = 0;
		      	for (int l = i + 1; l < n; l++) {
		        	matrix[k][l] += matrix[i][l] * temp;
		      	}
		      	result[k] += result[i] * temp;
		    }
	    }
  	}
  	for (int m = n - 1; m >= 0; m--) {
    	temp = result[m];
    	for (int i = n - 1; i > m; i--) {
    		temp -= ans.get(i) * matrix[m][i];
    	}
    	ans.set(m,temp / matrix[m][m]);
  	}
  	return ans;
}
\end{lstlisting}

\subsection{Greatest common divisor}
Calcula el m\'aximo com\'un divisor entre a y b mediante el algoritmo de Euclides

\begin{lstlisting}

int mcd(int a, int b) {
	int aux;
	while(b!=0){
		a %= b;
		aux = b;
		b = a;
		a = aux;
	}
	return a;
}
\end{lstlisting}

\subsection{Lowest Common multiple}
C\'alculo del m\'inimo com\'un m\'ultiplo usando el m\'aximo com\'un divisor REQUIERE mcd(a,b)

\begin{lstlisting}

int mcm(int a, int b) {
	return a*b/mcd(a,b);
}
\end{lstlisting}

\subsection{Prime Factorization}
Guarda en primeFactors la lista de factores primos del value de menor a mayor. IMPORTANTE: Debe ejecutarse primero la criba de Eratostenes.  La criba debe existir al menos hasta la raiz cuadrada de value (se  recomienda dejar un poco de excedente).

\begin{lstlisting}

import java.util.ArrayList;

static ArrayList<Long> primeFactors = new ArrayList<Long>();

static void calculatePrimeFactors(long value){
	primeFactors.clear();
	long temp = value;
	int factor;
	for (int i = 0; (long)primes.get(i) * primes.get(i) <= value; ++i){
		factor = primes.get(i);
		while (temp % factor == 0){
			primeFactors.add((long)factor);
			temp /= factor;
		}
	}
	if (temp != 1) {
		primeFactors.add(temp);
	}
}
\end{lstlisting}

\subsection{Sieve of Eratosthenes}
Guarda en primes los n\'umeros primos menores a MAX

\begin{lstlisting}

import java.util.ArrayList;

static int MAX = 10000000;
static ArrayList<Integer> primes = new ArrayList<Integer>();
static boolean sieve[] = new boolean[MAX+5];
	
static void calculatePrimes() {
  sieve[0] = sieve[1] = true;
  int i;
  for (i = 2; i * i <= MAX; ++i) {
    if (!sieve[i]) {
      primes.add(i);
      for (int j = i * i; j <= MAX; j += i)
        sieve[j] = true;
    }
  }
  for(; i <= MAX; i++){
  	if (!sieve[i]) {
      primes.add(i);
    }
  }
}
\end{lstlisting}

\section{String}

\subsection{KMP's Algorithm}

Encuentra si el string pattern se encuentra en el string cadena.
\begin{lstlisting}

import java.util.ArrayList;

static ArrayList<Integer> table(String pattern){
	int m=pattern.length();
	ArrayList<Integer> border = new ArrayList<Integer>();
	border.add(0);
	int temp;
	for(int i=1; i<m; ++i){
		border.add(border.get(i-1));
		temp = border.get(i);
		while(temp>0 && pattern.charAt(i)!=pattern.charAt(temp)){
			if(temp <= i+1){
				border.set(i,border.get(temp-1));
				temp = border.get(i);
			}
		}
		if(pattern.charAt(i) == pattern.charAt(temp)){
			border.set(i,temp+1);
		}
	}
	return border;
}

static boolean kmp(String cadena, String pattern){
	int n=cadena.length();
	int m=pattern.length();
	ArrayList<Integer> tab=table(pattern);
	int seen=0;

	for(int i=0; i<n; i++){
		while(seen>0 && cadena.charAt(i)!=pattern.charAt(seen)){
			seen=tab.get(seen-1);
		}
		if(cadena.charAt(i)==pattern.charAt(seen))
			seen++;
		if(seen==m){
			return true;
		}
	}
	return false;
}
\end{lstlisting}

\section{Tips and formulas}

\subsection{Catalan Number}

{\LARGE $C_{n} = \frac{1}{n+1}\binom{2n}{n} = \frac{(2n)!}{(n+1)!n!}$}\\ \\
Primeros 30 n\'umeros de Catal\'an:\\
\begin{tabular}{|l|l|}
\hline
n & $C_{n}$ \\ \hline
0 & 1 \\ \hline
1 & 1 \\ \hline
2 & 2 \\ \hline
3 & 5 \\ \hline
4 & 14 \\ \hline
5 & 42 \\ \hline
6 & 132 \\ \hline
7 & 429 \\ \hline
8 & 1.430\\ \hline
9 & 4.862\\ \hline
10 & 16.796\\ \hline
11 & 58.786\\ \hline
12 & 208.012\\ \hline
13 & 742.900\\ \hline
14 & 2.674.440\\ \hline
15 & 9.694.845\\ \hline
16 & 35.357.670\\ \hline
17 & 129.644.790\\ \hline
18 & 477.638. 700\\ \hline
19 & 1.767.263.190\\ \hline
20 & 6.564.120.420\\ \hline
21 & 24.466.267.020\\ \hline
22 & 91.482.563.640\\ \hline
23 & 343.059.613.650\\ \hline
24 & 1.289.904.147.324\\ \hline
25 & 4.861.946.401.452\\ \hline
26 & 18.367.353.072.152\\ \hline
27 & 69.533.550.916.004\\ \hline
28 & 263.747.951.750.360\\ \hline
29 & 1.002.242.216.651.368\\ \hline
30 & 3.814.986.502.092.304\\ \hline
\end{tabular}



\subsection{Euclidean Distance}
F\'ormula para calcular la distancia Euclideana entre dos puntos en el plano cartesiano (x,y). \\
Extendible a 3 dimensiones\\ \\
$ d_{E}(P_{1},P_{2}) = \sqrt{(x_{2}-x_{1})^{2}+(y_{2}-y_{1})^{2}} $




\subsection{Permutation and combination}

Combinaci\'on (Coeficiente Binomial): N\'umero de subconjuntos de k elementos escogidos de un conjunto con n elementos\\ \\
{\LARGE $ \binom{n}{k} = \binom{n}{n-k} = \frac{n!}{k!(n-k)!} $}\\

Combinaci\'on con repetici\'on: N\'umero de grupos formados por n elementos, partiendo de m tipos de elementos.\\ \\


{\LARGE $ CR_{m}^{n} = \binom{m+n-1}{n} = \frac{(m + n - 1)!}{n!(m-1)!} $}\\

Permutaci\'on: N\'umero de formas de agrupar n elementos, donde importa el orden y sin repetir elementos\\ \\

{\LARGE $ P_{n} = n! $}\\

Elegir r elementos de n posibles con repetici\'on\\ \\

{\LARGE $ n^{r} $}\\

Permutaciones con repetici\'on: Se tienen n elementos donde el primer elemento se repite a veces , el segundo b veces , el tercero c veces, ...\\ \\

{\LARGE $ PR_{n}^{a,b,c...} = \frac{P_{n}}{a!b!c!...}$}\\

Permutaciones sin repetici\'on: N\'umero de formas de agrupar r elementos de n disponibles, sin repetir elementos \\ \\

{\LARGE $ \frac{n!}{(n-r)!}$}\\
}
\subsection{Time Complexities}

Aproximaci\'on del mayor n\'umero n de datos que pueden procesarse para cada una de las complejidades algoritmicas. Tomar esta tabla solo como referencia.\\ 

\begin{tabular}{|l|l|}
\hline
Complexity & n\\ \hline
$O(n!)$ & 11\\ \hline
$O(n^{5})$ & 50\\ \hline
$O(2^{n}*n^{2})$ & 18\\ \hline
$O(2^{n}*n)$ & 22\\ \hline
$O(n^{4})$ & 100\\ \hline
$O(n^{3})$ & 500\\ \hline
$O(n^{2}\log_{2}n)$ & 1.000\\ \hline
$O(n^{2})$ & 10.000\\ \hline
$O(n\log_{2}n)$ & $10^{6}$\\ \hline
$O(n)$ & $10^{8}$\\ \hline
$O(\sqrt{n})$ & $10^{16}$\\ \hline
$O(\log_{2}n)$ & -\\ \hline
$O(1)$ & -\\ \hline
\end{tabular}


\subsection{mod: properties}

\begin{enumerate}
\item (a \% b) \% b = a \% b (Propiedad neutro)
\item (ab) \% c = ((a \% c)(b \% c)) \% c (Propiedad asociativa en multiplicaci\'on)
\item (a + b) \% c = ((a \% c) + (b \% c)) \% c (Propiedad asociativa en suma)
\end{enumerate}
\end{document}
